%!TEX TS-program = xelatex
\documentclass[12pt]{letter}
\usepackage{xeCJK}
\usepackage{ruby}
\setCJKmainfont{Hiragino Kaku Gothic ProN}
\setmainfont{Hoefler Text}

\usepackage{setspace}
\doublespace

%
\usepackage[landscape,margin=0.5in]{geometry}
%%
\makeatletter
\newcommand{\defaultrubysep}{-1.7ex}
\renewcommand{\rubysep}{\defaultrubysep}
\newcommand{\english}{}
\renewcommand{\english}{-4.8ex}
\newlength{\kanji@kana@no@nagasa}
\newcommand{\eruby}[3]{%
    \ruby{\textbf{#1}}{#2}\renewcommand{\rubysep}{\english}%
    \settowidth{\kanji@kana@no@nagasa}{#1}%
    \hspace*{\dimexpr-\kanji@kana@no@nagasa\relax}%+0.5\@tempdimb
    \ruby{\phantom{#1}}{#3}\renewcommand{\rubysep}{\defaultrubysep}%
}
\makeatother
%%

\pagestyle{empty}
\begin{document}

\title{たんぽぽの旅}
\eruby{田}{た}{rice field}んぼの \eruby{畦道}{あぜみち}{footpath}に、\eruby{タンポポ}{tanpopo}{dandelion}が \eruby{咲いていました}{saiteimashita}{to bloom}。
Tanbo no azemichi, tanpopo saiteimashita.
\begin{description}
	\item[さく【咲く】] To bloom
	\item[たんぼ【田んぼ】] Rice field
	\item[あぜみち【畦道】] A walkway in the rice field
	\item[タンポポ] A dandelion
\end{description}
On a walkway in the ricefield, a dandelion blooms.
\newline
\linebreak[2]
\eruby{甘い}{あまい}{sweet}\eruby{匂い}{におい}{odor}のする\eruby{真っ}{ま}{emph.}\eruby{黄色な}{きろな}{yellow}\eruby{花}{はな}{flower}です。
Amai nioi no suru makkirona hana desu.
\begin{description}
	\item[あまい【あま】] Sweet
	\item[あまい【匂い】] Odor
	\item[まっ【真っ〜〜】] Emphasis modifier
	\item[きいろな【黄色い】] Yellow
	\item[はな【花】] Flower
\end{description}
It is a really yellow flower that smells sweet.
\newpage
\eruby{そこへ}{soko-e}{there} \eruby{ひらひら}{}{flutter} と \eruby{白い}{しろい}{white} \eruby{チョウチョ}{ch\={o}cho}{butterfly}が あそびに きて、「\ruby{タンポポ}{tanpopo}さん、あなたは \eruby{春}{はる}{spring}の \eruby{ブローチ}{bur\={o}chi}{brooch}ね。」 といいました。
Soko-e hirahira to shiroi ch\={o}ch\={o} ga asobi ni kite, "tanpopo-san, anata wa haru no buroochi-ne" to iimashita.
\begin{description}
	\item[そこへ] There
	\item[ひらひら] Flutter
	\item[しろい【白い】] White
	\item[ちょうちょ] Butterfly
	\item[あそびにきる] To come over, to visit
\end{description}
A white fluttery butterfly came over, "Mr. Dandelion you are a spring's brooch, aren't you?", it said.
\newline
\linebreak[2]
\eruby{白い}{しろい}{white} \eruby{チョウ}{ch\={o}}{butterfly}に そう \eruby{言}{え}{said}われると、 「\eruby{本当}{ほんとう}{truly}に そう 見えて。」 と、 タンポポも うれしく なりました。\newline
Shiroi ch\={o} ni s\={o} iwareru to: "Honto no s\={o} miete." to, tanpopo ureshiku narimashita.

\newpage
%今度は、ヒバリが あそびに きて、「タンポポさん、 あなたは のはらの おほしさまみたい。」 といいました。
%\hrule
%ヒバリに こう いわれると、 「まあ、 おほしさまだなんで。」 と うれしく なりました。
%\hrule
\end{document}